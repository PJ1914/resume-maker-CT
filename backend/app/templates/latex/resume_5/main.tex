%%%%%%%%%%%%%%%%%
% This is an sample CV template created using altacv.cls
% (v1.1.4, 27 July 2018) written by LianTze Lim (liantze@gmail.com). Now compiles with pdfLaTeX, XeLaTeX and LuaLaTeX.
% 
%% It may be distributed and/or modified under the
%% conditions of the LaTeX Project Public License, either version 1.3
%% of this license or (at your option) any later version.
%% The latest version of this license is in
%%    http://www.latex-project.org/lppl.txt
%% and version 1.3 or later is part of all distributions of LaTeX
%% version 2003/12/01 or later.
%%%%%%%%%%%%%%%%

%% If you need to pass whatever options to xcolor
\PassOptionsToPackage{dvipsnames}{xcolor}

%% If you are using \orcid or academicons
%% icons, make sure you have the academicons 
%% option here, and compile with XeLaTeX
%% or LuaLaTeX.
% \documentclass[10pt,a4paper,academicons]{altacv}

%% Use the "normalphoto" option if you want a normal photo instead of cropped to a circle
% \documentclass[10pt,a4paper,normalphoto]{altacv}

\documentclass[10pt,a4paper]{altacv}
%% AltaCV uses the fontawesome and academicon fonts
%% and packages. 
%% See texdoc.net/pkg/fontawecome and http://texdoc.net/pkg/academicons for full list of symbols.
%% 
%% Compile with LuaLaTeX for best results. If you
%% want to use XeLaTeX, you may need to install
%% Academicons.ttf in your operating system's font 
%% folder.

% Change the page layout if you need to
\geometry{left=1cm,right=9cm,marginparwidth=6.8cm,marginparsep=1.2cm,top=1.25cm,bottom=1.25cm,footskip=2\baselineskip}

% Change the font if you want to.

% If using pdflatex:
\usepackage[T1]{fontenc}
\usepackage[utf8]{inputenc}
% \usepackage[default]{lato}
\usepackage{helvet}
\renewcommand{\familydefault}{\sfdefault}

% If using xelatex or lualatex:
% \setmainfont{Lato}

% Change the colours if you want to
\definecolor{Navy}{HTML}{000080}
\definecolor{SlateGrey}{HTML}{2E2E2E}
\definecolor{LightGrey}{HTML}{666666}
\colorlet{heading}{Navy}
\colorlet{accent}{Navy}
\colorlet{emphasis}{SlateGrey}
\colorlet{body}{LightGrey}


% Change the bullets for itemize and rating marker
% for \cvskill if you want to
\renewcommand{\itemmarker}{{\small\textbullet}}
\renewcommand{\ratingmarker}{\faCircle}
%% sample.bib contains your publications
%\addbibresource{sample.bib}

\usepackage[colorlinks]{hyperref}
\usepackage{filecontents}

\begin{filecontents*}{page1sidebar.tex}
\BLOCK{ if skills }
\cvsection{Skills}
\BLOCK{ for skill in skills }
\cvtag{\VAR{skill.category|escape_tex}: \VAR{skill.items|join(', ')|escape_tex}}
\BLOCK{ endfor }
\divider
\BLOCK{ endif }

\BLOCK{ if certifications }
\cvsection{Certifications}
\BLOCK{ for cert in certifications }
\cvevent{\VAR{cert.name|escape_tex}}{\VAR{cert.issuer|escape_tex}}{\VAR{cert.date|escape_tex}}{}
\divider
\BLOCK{ endfor }
\BLOCK{ endif }

\BLOCK{ if languages }
\cvsection{Languages}
\BLOCK{ for lang in languages }
\cvtag{\VAR{lang.language|escape_tex} (\VAR{lang.proficiency|escape_tex})}
\BLOCK{ endfor }
\divider
\BLOCK{ endif }

\BLOCK{ if achievements }
\cvsection{Achievements}
\begin{itemize}
\BLOCK{ for ach in achievements }
\item \textbf{\VAR{ach.title|escape_tex}} (\VAR{ach.date|escape_tex})
\BLOCK{ endfor }
\end{itemize}
\BLOCK{ endif }

\end{filecontents*}

\begin{document}

\name{\VAR{full_name|escape_tex}}
\tagline{\VAR{summary|truncate(50)}}
\personalinfo{%
  % Not all of these are required!
  % You can add your own with \printinfo{symbol}{detail}
  \email{\VAR{email|escape_tex}}
  \phone{\VAR{phone|escape_tex}}
%  \homepage{www.homepage.com}
%  \twitter{@twitterhandle}
  \linkedin{\VAR{linkedin|escape_tex}}
  %% You MUST add the academicons option to \documentclass, then compile with LuaLaTeX or XeLaTeX, if you want to use \orcid or other academicons commands.
%   \orcid{orcid.org/0000-0000-0000-0000}
}

%% Make the header extend all the way to the right, if you want. 
\begin{fullwidth}
\makecvheader
\end{fullwidth}

%% Depending on your tastes, you may want to make fonts of itemize environments slightly smaller
% \AtBeginEnvironment{itemize}{\small}


%% Provide the file name containing the sidebar contents as an optional parameter to \cvsection.
%% You can always just use \marginpar{...} if you do
%% not need to align the top of the contents to any
%% \cvsection title in the "main" bar.
\cvsection[page1sidebar]{Experience}

\BLOCK{ for exp in experience }
\cvevent{\VAR{exp.position|escape_tex}}{\VAR{exp.company|escape_tex}}{\VAR{exp.start_date|escape_tex} -- \VAR{exp.end_date|escape_tex}}{\VAR{exp.location|escape_tex}}
\begin{itemize}
\BLOCK{ if exp.description }
\item \VAR{exp.description|escape_tex}
\BLOCK{ endif }
\BLOCK{ for highlight in exp.highlights }
\item \VAR{highlight|escape_tex}
\BLOCK{ endfor }
\end{itemize}
\divider
\BLOCK{ endfor }

\cvsection{Education}

\BLOCK{ for edu in education }
\cvevent{\VAR{edu.degree|escape_tex} in \VAR{edu.field|escape_tex}}{\VAR{edu.institution|escape_tex}}{\VAR{edu.start_date|escape_tex} -- \VAR{edu.end_date|escape_tex}}{\VAR{edu.location|escape_tex}}
\BLOCK{ if edu.gpa }
GPA: \VAR{edu.gpa|escape_tex}
\BLOCK{ endif }
\divider
\BLOCK{ endfor }

\BLOCK{ if projects }
\cvsection{Projects}
\BLOCK{ for proj in projects }
\cvevent{\VAR{proj.name|escape_tex}}{Technologies: \VAR{proj.technologies|join(', ')|escape_tex}}{}{}
\BLOCK{ if proj.description }
\VAR{proj.description|escape_tex}
\BLOCK{ endif }
\BLOCK{ if proj.link }
\VAR{proj.link|escape_tex}
\BLOCK{ endif }
\divider
\BLOCK{ endfor }
\BLOCK{ endif }

\end{document}
